\documentclass{beamer}
%\usepackage{pgfpages}
%\pgfpagesuselayout{4 on 1}[a4paper,border shrink=5mm]
\usepackage[utf8]{inputenc}
\usepackage[serbianc]{babel}
\usepackage{textgreek}
\usepackage{pgfpages}

\usetheme{Boadilla}
\usecolortheme[named=violet]{structure}
\usepackage{caption}
\usepackage{subcaption}
\usepackage[final]{pdfpages}
\usepackage{xcolor}

\usepackage{listings}



\useoutertheme[footline=authortitle]{miniframes}

\title[Програмски језик \textit{Python} - основи]{\color{black}\smallУвод у \\\color{violet}\normalsizeПрограмски језик \textit{Python}}
\vspace{-4cm}\author{Александар Пајкановић}
\titlegraphic{\includegraphics[width=2cm]{logo}}

%\titlegraphic{\includegraphics[width=2cm]{logo}%\hspace{10.75cm}%

%}
\institute{\texttt{nanoluka.org}\\
YouTube.com/nanoluka\\
instagram.com/nanolukaorg\\
twitter.com/nanolukaorg\\
github.com/nanoluka\\
\texttt{nanolukaorg@gmail.com}}

\date{\color{violet}Октобар 2020}

\begin{document}

\maketitle

\begin{frame}{Кориштење овог документа}
\begin{itemize}
    \item \textbf{Лиценца}: 
    \begin{itemize}
        \item Документ и пратеће к\^{о}дове објављујем под лиценцом \color{violet}\href{https://creativecommons.org/licenses/by-sa/4.0//legalcode}{\textit{Attribution-ShareAlike — CC BY-СА}}\color{black}. Ова лиценца дозвољава ремикс и прераду, као и комерцијално кориштење дјела, ако/док се \textbf{правилно назначава име аутора} и ако се \textbf{прерада лиценцира под истим условима}. Ова лиценца се често упоређује са „\textit{copyleft}” лиценцирањем слободног софтвера или софтвера отвореног к\^{о}да. Сва дјела настала на основу дјела лиценцираног овом лиценцом, требало би да буду лиценцирана истом лиценцом, која, поред осталог, дозвољава комерцијално кориштење.
    \end{itemize}
    \item \textbf{Навођење (цитирање)}: 
    \begin{itemize}
        \item Ако сте користили овај документ као извор у сопственим материјалима, молим да цитирате на сљедећи начин: А. Пајкановић, ,,Увод у програмски језик \textit{C} кроз практичне примјере'', доступно на \color{violet}\href{github.com/nanoluka/python.git}{github.com/nanoluka/python.git}
    \end{itemize}
\end{itemize}
\end{frame}



\begin{frame}{Програмирање и програмски језици}
\begin{itemize}
    \item Рачунар у општем смислу је машина способна да изврши било који алгоритам пратећи дефинисан скуп правила - ткз. Тјурингова машина (теоријска замисао).
    \item Алгоритам је недвосмислен метод рјешавања конкретног проблема
    \item Рачунар се већ читав вијек састоји од процесора и меморије - ткз. Фон Нојманова архитектура (реализација)
    \item Програмирање, у практичном смислу, можемо рећи да представља састављање текстуалног (понегдје и графичког) описа жељеног понашања рачунара у смислу обраде корисничких података и представљања резултата тог процеса кориснику
    \item Програмски језик је скуп инструкција (кључних ријечи) које, ако су посложене у складу са унапријед дефинисаним правилима, рачунар може да прихвати - при чему ово ,,прихвати'' значи да је у стању да, узев дате податке корисника на улазу, кориснику прикаже очекивани излаз
\end{itemize}
\end{frame}

\begin{frame}{Програмски језик \textit{Python}}
\begin{itemize}
    \item Постоји много подјела програмских језика, најважнија је на:
    \begin{itemize}
        \item језике вишег нивоа, и
        \item језике нижег нивоа
    \end{itemize}
    \item \textit{Python} спада у језик вишег нивоа. Генерално, сви програмски језици за које сте чули су, у ствари, у истој групи.
    \item Друга важна подјела је према парадигми којој програмски језик припада, тренутно су актуелне четири парадигме (које се, често, међусобно пресјецају):
    \begin{itemize}
        \item императивно програмирање,
        \item функционално,
        \item логичко, и
        \item објектно-оријентисано.
    \end{itemize}
    \item У нашем данашњем језику можемо да пишемо програме који припадају двјема парадигмама - функционалном и објектно-оријентисаном програмирању
    \item Ми, пак, данас нећемо користити ни једно ни друго
\end{itemize}
\end{frame}

\begin{frame}{Програмски језик \textit{Python}}
\begin{itemize}
    \item Иначе, \textit{Python} је 1991. објавио Гвидо ван Росум
    \item С обзиром да смо прошли пут причали о језику \textit{C}, ево неколико особина \textit{Python vs. C}:
    \begin{itemize}
        \item лакше га је савладати, синтакса је корисницима читљивија
        \item \textit{Python} је интерпретирани језик, што значи да се једна по једна линија к\^{о}да преводе, док се \textit{C} програм компајлира. Као посљедица, извршавање \textit{Python} програма је спорије
        \item иста чињеница узрокује лакше уочавање грешака
        \item варијабле се не декларишу експлицитно, него имплицитно приликом прве употребе
        \item нема показивача (напредна функционалност језика \textit{C}, нисмо се тиме бавили у другом предавању, наводим само потпуности ради
        \item аутоматско рјешавање потенцијалних проблема са меморијом
        \item неупоредиво већа подршка најразличитијих библиотека и пакета у свим могућим доменима науке, технологије и живота уопште
    \end{itemize}
\end{itemize}
\end{frame}

\begin{frame}{Програмски језик \textit{Python}}
\begin{itemize}
    \item Укратко, поступак од идеје до реализације састоји се од четири корака:
    \begin{enumerate}
        \item алгоритам,
        \item псеудо к\^{о}д,
        \item изворни к\^{о}д - овдје се пише синтакса (\texttt{.py}),
        \item и сад имамо три могућности:
        \begin{itemize}
            \item извршавање појединачних линија, директно из терминала,
            \item извршавање у виду скрипте, и
            \item превод у извршну датотеку (\texttt{.exe}) $\leftarrow$ нећемо радити
        \end{itemize}
    \end{enumerate}
    \item На том путу, сусрећемо се са разним невољама, а подијелићемо их у грешке:
    \begin{itemize}
        \item синтаксичке, и
        \item семантичке.
    \end{itemize}
    \item Ова презентација не представља потпуне информације о рачунарима, програмирању и програмском језику \textit{Python}, него служи као увод или преглед, како бисмо се упознали са основама и знали гдје и шта да тражимо како бисмо стварно научили.
\end{itemize}
\end{frame}

\begin{frame}{Потребни алати}
\begin{itemize}
    \item Рачунар - бар током овог предавања, али може и телефон или таблет - само је унос проблематичан
    \item \textit{Windows} оперативни систем - може, наравно, и неки други али данас рад демонстрирамо овако
    \item \textit{Python} има двије актуелне верзије (2.7 и 3), с тим да се већ годинама упорно труде да потисну 2.7 - па је треба избјегавати за нове пројекте
    \item \textit{Python} као апликација за \textit{Windows} бесплатно је доступан на \color{violet}\href{python.org}{https://www.python.org}\color{black}.
    %\item \textit{Microsoft Visual Studio} је потребан за примјере седам и осам доступан на: \color{violet}\href{https://visualstudio.microsoft.com/downloads}{https://visualstudio.microsoft.com/downloads}\color{black}
    \item Познавање енглеског језика
    \item Добра воља, чврста одлука да се не одустаје и упоран рад
\end{itemize}
\end{frame}

\begin{frame}[fragile]{Примјер 1 - Испис текста}

\begin{itemize}
\item \textit{File} $\rightarrow$ \textit{New File}
\end{itemize}

\begin{block}{HelloICBL}
\begin{lstlisting}
print("Hello ICBL!")
\end{lstlisting}
\end{block}

\begin{itemize}
\item \textit{File} $\rightarrow$ \textit{Save}
\item \textit{Run} $\rightarrow$ \textit{Run Module}
\end{itemize}

\begin{block}{Output}
\begin{lstlisting}
Hello ICBL!
>>> 
\end{lstlisting}
\end{block}

\end{frame}

\begin{frame}{Типови података}
\begin{itemize}
    \item Основна подјела сигнала:
    \begin{itemize}
        \item аналогни - звук, бежични пренос података, зрачење, слика
        \item дигитални - запис аналогних, али користећи само ограничен број нивоа - ако су само два нивоа, онда су то бинарни сигнали
    \end{itemize}
    \item Данашњи комерцијално доступни рачунари су дигитални и разумију искључиво бинарне податке
    \item Математички запис је дат прије два вијека, данас познат као бинарна или Булова алгебра
    \item Једна бинарна цифра назива се бит (енгл. \textit{binary digit} $\rightarrow$ \textit{bit})
    \item Осам бита је бајт (енгл. \textit{byte}), kilo, mega, итд.
    \item Подаци су осмобитни, 32-битни, итд.
    \item Корисни записи су још и октални и хексадецимални
    \item Децимални број 6, осмобитно се представља као \texttt{00000110}, број 126 пишемо \texttt{01111110}, а -126 je \texttt{10000010}
    \item Означени и неозначени
\end{itemize}
\end{frame}

\begin{frame}[fragile]{Примјер 2 - Типови података}
\begin{block}{DataTypes}
\begin{lstlisting}
x = 5
print(type(x))
x = "Hello World"
x = 20
x = 20.5
x = ["apple", "banana", "cherry"]
x = ("apple", "banana", "cherry")
x = range(6)
x = {"name" : "John", "age" : 36}
x = {"apple", "banana", "cherry"}
x = True

\end{lstlisting}
\end{block}
\end{frame}

\begin{frame}[fragile]{Примјер 2 - Типови података}
\begin{block}{Output}
\begin{lstlisting}
<class 'int'>
<class 'str'>
<class 'int'>
<class 'float'>
<class 'list'>
<class 'tuple'>
<class 'range'>
<class 'dict'>
<class 'set'>
<class 'bool'>
>>> 
\end{lstlisting}
\end{block}
\end{frame}

\begin{frame}[fragile]{Примјер 3 - Аритметичке и логичке операције}
\begin{block}{ALU}
\begin{lstlisting}
a = 2
b = 4
c = 'a'
d = 6.2
print("zbir: a+b = " + str(a+b))
print("eksponent: a**b = " + str(a**b))
print("proizvod: a*c = " + str(a*c))
print("kolicnik - cjelobrojno: a//b = " + str(a//b))
print("kolicnik - decimalno: a/d = " + str(a/d))
print("jednakost: a == b = " + str(a==b))
print("binarno I: a & b = " + str(a & b))
print("logicko I: a and b = " + str(a and b))
\end{lstlisting}
\end{block}
\end{frame}

\begin{frame}{Преклапање оператора, кастовање}
    \begin{itemize}
        \item Када помножимо два броја, добијемо производ
        \item Шта ће бити ако помножимо број и слово?
        \item Или број и ријеч?
        \item Претварање једног типа податка у други назива се кастовање
        \item Објекти:
        \begin{itemize}
            \item промјенљивих вријености - \textit{mutable}
            \item непромјенљивих вриједнсоти - \textit{immutable}
        \end{itemize}
    \end{itemize}
\end{frame}

\begin{frame}{Контрола т\^{о}ка}
\begin{itemize}
    \item Рачунари не одлучују. Бар не још увијек
    \item Рачунари поступају по инструкцијама
    \item Инструкција може да се изврши или не изврши у зависности од услова, односно може да се изврши једна или друга, опет, наравно, зависно од стања нечег другог:
    \begin{itemize}
        \item ако је температура већа од 22$^o$ C, искључи гријање
        \item ако је температура мања од 18$^o$ C, укључи гријање
        \item не извршавај ништа, док се не притисне овај тастер
        \item изврши сабирање свих бројева у овом низу, тј. 10 000 сабирања
    \end{itemize}
    \item Графички приказ т\^{о}ка извршавања назива се дијаграм т\^{о}ка
    \item Т\^{о}к контролишемо користећи се гранањем (\texttt{if-else, switch}) и петљама (\texttt{for, while})
\end{itemize}
\end{frame}

\begin{frame}[fragile]{Примјер 4 - Унос података}
\begin{block}{DataInput}
\begin{lstlisting}
x = input("prvi broj: ")
y = input("drugi broj: ")
x = int(x)
y = int(y)
z = x + y
print("Zbir je: " + str(z))
\end{lstlisting}
\end{block}
\end{frame}

\begin{frame}[fragile]{Примјер 5 - Гранање}
\begin{block}{Branch}
\begin{lstlisting}
x = input("prvi broj: ")
r = input("radnja: ")
y = input("drugi broj: ")
x = int(x)
y = int(y)
if r == "+": 
    print("zbir: " + str(x+y))
elif r == "-": print("razlika: " + str(x-y))
elif r == "*": print("proizvod: " + str(x*y))
elif r == ":": print("kolicnik: " + str(x/y))
else: print("Neispravna radnja!")
\end{lstlisting}
\end{block}
\end{frame}

\begin{frame}[fragile]{Примјер 6 - Петља}
\begin{block}{Loop}
\begin{lstlisting}
while 1:
    x = input("prvi broj: ")
    r = input("radnja: ")
    y = input("drugi broj: ")
    print("\n\ntreba da izracunamo: " + x + r + y )
    if r == "+": print("zbir: " + str(x+y))
    elif r == "-": print("razlika: " + str(x-y))
    elif r == "*": print("proizvod: " + str(x*y))
    elif r == ":": print("kolicnik: " + str(x/y))
    else: print("Neispravna radnja!")
    izlaz = input("Kraj?\t")
    if izlaz == "d" or izlaz == "D":
        break
\end{lstlisting}
\end{block}
\end{frame}

\begin{frame}{Енкапсулација и апстракција}
\begin{itemize}
    \item Тако далеко сам видио, зато што сам стојао сам на плећима дивова
    \item Све што постоји, а да је створено људском руком, настало је комбинацијом већ постојећег
    \item Зашто да измишљамо топлу воду?
    \item Дајте ми ослонац, помјерићу свијет
    \item Право питање је: шта је проблем?
    \item Видјели смо који су типови података, видјели смо које су доступне аритметичке операције.
    \item Како ћемо израчунати коријен? Синус? Интеграл? 
    \item Управљати роботом?
    \item Програмирање игара?
\end{itemize}
\end{frame}

\begin{frame}[fragile]{Библиотеке}
\begin{itemize}
    \item \texttt{numpy} - нумерички прорачуни
    \item \texttt{matplotlib} - графичка представа података
    \item \texttt{math} - напредније математичке операције
    \item \texttt{sys} - за рад са системским ресурсима
    \item \texttt{pytorch} - машинско учење
    \item \texttt{scipy} - још напредније математичке операције
    \item \texttt{pyserial} - комуникација преко серијског порта
    \item \texttt{random} - случајне вриједности
    \item \texttt{time} - в
    реално вријеме
    \item \texttt{pyfirmata} - комуникација серијским портом са Ардуином
    \item и бесконачно много других...
    \item кључна ријеч \texttt{import}
    \item \textit{GNU/Linux} оперативни системи се много лакше сналазе са свим овим ресурсима
\end{itemize}
\end{frame}

\begin{frame}[fragile]{Примјер 7 - Кориштење библиотека и функција}
\begin{block}{Function}
\begin{lstlisting}
import random

broj = input("Koliko kockica?\n")
broj = int(broj)

ponovo = "d"
while ponovo == "d" or ponovo == "D":
    print("Bacam kockicu...")
    for i in range(broj):
        print("Kockica " + str(i+1) + ": " + str(random.randint(1,6)))
    ponovo = input("Ponovo?\n")
\end{lstlisting}
\end{block}
\end{frame}

\begin{frame}[fragile]{Примјер 8 - Писање и функција}
\begin{block}{Function}
\begin{lstlisting}
def identitet(ime, prezime):
  print(ime + " " + prezime)
  x = 5

identitet("Nikola", "Tesla")
x = 6
print(x)
\end{lstlisting}
\end{block}

\begin{itemize}
    \item Домен вриједности промјенљиве:
    \begin{itemize}
        \item локални, и
        \item глобални
    \end{itemize}
\end{itemize}
\end{frame}

\begin{frame}{Уграђени (енгл. \textit{embedded}) системи}
\begin{itemize}
    \item Уграђени системи су рачунарски системи намијењени за извршавање специфичне функције у реалном времену, и могу да буду сасставни дио обимнијих електромеханичких (над)система.
    \item Рачунари су, на примјер, системи опште намјене, дакле нису ,,уграђени'' у овом смислу.
    \item Махом су дигитални, а централни дио је увијек микроконтролер.
    \item Микроконтролер је процесор са периферијама и (малом) меморијом.
    \item \textit{Internet of Things} - интернет свега, паметни системи
    \item Аутомобилска индустрија (енгл. \textit{automotive})
    \item Интересантан свијет, на граници између хардвера и софтвера, рачунарска електроника, рачунарски инжењеринг, мехатроника, роботика.
    \item Ардуино је сјајан први корак на том путу.
\end{itemize}
\end{frame}

\begin{frame}[fragile]{Примјер 9 - Трепћући Ардуино}
\begin{block}{Blink}
\begin{lstlisting}
import pyfirmata
import time

board = pyfirmata.Arduino('COM3')

while True:
    board.digital[13].write(1)
    time.sleep(1)
    board.digital[13].write(0)
    time.sleep(1)
\end{lstlisting}
\end{block}
\end{frame}

\begin{frame}{Примјер 10 - Игре}
\begin{itemize}
    \item Једноставније игре, текстуалне, као што је бацање коцкице малоприје већ можете да развијате, имате све што је потербно - питање је само идеје.
    \item \color{violet}\href{https://www.coursera.org/learn/interactive-python-1}{An Introduction to Interactive Programming in Python}
    \item \href{http://www.codeskulptor.org/}{CodeSkulptor}
    \item \textit{Demos:}
    \begin{itemize}
        \item \textit{Galaxy Invaders}
        \item \textit{Pyman}
        \item \textit{Rice Racer}
    \end{itemize}
\end{itemize}
\end{frame}

\begin{frame}{Референце}
\begin{itemize}
    \item \color{violet}\href{https://www.google.com/}{Google}!!!
    \item \href{python.org}{Python}
    \item \color{violet}\href{https://skolakoda.github.io/kursevi/}{Школа к\^{o}да}, \color{black} 
    \item \color{violet}\href{https://www.learnpython.org/}{LearnPython}
    \item \color{violet}\href{https://www.coursera.org/}{Coursera} \color{black}и \color{violet}\href{https://www.edx.org/}{edX}
    \item \color{violet}\href{http://uticnionica.etf.unibl.org/}{Утичнионица}\color{black} - основно о Ардуину, са Електротехничког факултета у Бањој Луци
    %\item Упутство за рад са \textit{CNC} наставком за Ардуино, \color{violet}\href{https://www.youtube.com/watch?v=TMK_fLgpESQ}{видео} \color{black}и \color{violet}\href{https://www.electroniclinic.com/arduino-cnc-shield-v3-0-and-a4988-hybrid-stepper-motor-driver-joystick/}{текст}
    \item \color{violet}\href{http://1000zabuducnost.org/}{1000 за будућност}\color{black} - увод у рад са:
    \begin{itemize}
        \item Пи рачунар
        \item програмски језик \textit{Python}
    \end{itemize}
    \item Званична презентација \color{violet}\href{https://www.raspberrypi.org/}{\textit{Raspberry Pi}}\color{black}
    \item Увод у електронику - серија \color{violet}\href{https://www.youtube.com/playlist?list=PLh-StTZA7RZ6Ch6Esin2mnoPzFvLZOYu3}{Три минута електронике}
    \item \href{https://www.nanoluka.org}{Књига о LTspice}\color{black}, софтверском симулатору електричних кола
\end{itemize}
\end{frame}

\end{document}